\documentclass[DM,authoryear,toc]{lsstdoc}
% lsstdoc documentation: https://lsst-texmf.lsst.io/lsstdoc.html
\input{meta}

% Package imports go here.
\usepackage{hyperref}


% Local commands go here.
\newcommand{\cvd}{COVID-19\,}
\newcommand{\qa}{Q\&A\,}
\newcommand{\raw}{Rubin Algorithms Workshop\,}
\newcommand{\rubin}{Rubin Observatory\,}
\setcounter{secnumdepth}{2}


%If you want glossaries
%\input{aglossary.tex}
%\makeglossaries

\title[Virtual Rubin Algorithms Workshop]{Virtual Rubin Algorithms Workshop.}

% Optional subtitle
% \setDocSubtitle{A subtitle}

\author{%
Leanne Guy
}

\setDocRef{DMTN-146}
\setDocUpstreamLocation{\url{https://github.com/lsst-dm/dmtn-146}}

\date{\vcsDate}

% Optional: name of the document's curator
% \setDocCurator{The Curator of this Document}

\setDocAbstract{%
% \footnote{\url{https://ls.st/law}\label{law}}
In March 2020, Rubin Observatory hosted a workshop on the topic of image processing algorithms for the Legacy Survey of Space and Time (LSST). (\cite{RAW2020}). 
The workshop was originally planned as an in-person event of about 80 people at Princeton University. 
Two weeks prior to the planned start date, in light of the uncertainty around the evolving coronavirus disease 2019 (\cvd), and with the health and safety of all attendees first and foremost in mind, the SOC took the decision to convert the workshop to an all-virtual format. 
This technical note describes how we modified and ran that all-virtual workshop, and  provides advice on how to organize similar such  online workshops, meetings and conferences.
}

% Change history defined here.
% Order: oldest first.
% Fields: VERSION, DATE, DESCRIPTION, OWNER NAME.
% See LPM-51 for version number policy.
\setDocChangeRecord{%
  \addtohist{1}{2020-03-24}{First draft.}{Leanne Guy}
}


\begin{document}

% Create the title page.
\maketitle
% Frequently for a technote we do not want a title page  uncomment this to remove the title page and changelog.
% use \mkshorttitle to remove the extra pages
% \mkshorttitle 
% ADD CONTENT HERE
% You can also use the \input command to include several content files.

% % % % % % % % % % % % % % % % % % % % % % % % % % % % % % % % %
\section{Introduction} 

In March 2020, Rubin Observatory planned to host a workshop on the topic of image processing algorithms for the Legacy Survey of Space and Time (LSST) (\cite{RAW2020}) for about 80 people at Princeton University.  
Given the increasing health risks and anxiety associated with the outbreak of the coronavirus disease 2019 (\cvd) at the time, the SOC felt strongly that the only responsible course of action was to not hold a primarily in-person based workshop. 
Rather than canceling or postponing the event, the workshop was converted to an all-virtual format. Holding a virtual online workshop rather than cancelling allowed us to still achieve many of the goals of the  original in-person workshop.

%%%%%%%%% The workshop format 	 %%%%%%%%%%%%%%%%%%%%%%%%
\section{Workshop format} 
One of the biggest challenge for any all-virtual meeting is accommodating the many time-zones across which participants are located.
\raw participants were located across all US time zones, Europe, Australia and Asia.  
This was the  strongest factor in redesigning the workshop for an all-virtual format was the need to accommodate many time zones. 
The best compromise that allowed us to accommodate all US and European timezones in a tolerable time-friendly block,  we shortened the day from 8hrs as planned for the in-person workshop to 5.5 hours.  Unfortunately this was was not a time-friendly for Asia-Pacific time zones, but it is not possible to accommocate the world and this was the best possible)

\section{Workshop Infrastructure} 
A combination of several online collaboration tools were used to make this workshop a success

\paragraph{Zoom}
Zoom was chosen as the platform for the virtual workshop. 

\paragraph{Slack}

\paragraph{Google drive}
Google drive was used to share documents with all the workshop participants.  "Participants Shared Drive"  to share essential information with all participants in one place. 
This included all the presentations and any supporting information, the Zoom connection details, the agenda and the documents for taking live notes. 
The live notes were set up in advance and all participants were invited to contribute.  

%% 
\paragraph{Website}
The \raw website, \footnote{\url{https://ls.st/law}\label{law}} was used for registration, to convey  organizational information, especially ongoing updates related to the \cvd situation and to serve the archived presentations and recordings and recordings to the public after the event. It was provided by the Rubin communications team. 

%%%% Running the workshop 
\section{Before the workshop}

\section{During the workshop}



\paragraph{Roles and responsibilities}
\paragraph{Controlling the workshop flow}


\section{After the meeting}

A feedback survey was sent out to all participants following the workshop to solicit feedback on what did and did not work.  Advice received has been incorporated into this technote.  Specific feedback was sought on how well the virtual experience worked and ways to improve it, both from the techniology standpoint and the tools, meeting procedures and processes. 

 the choice, format and logistics of an all-virtual workshop as well as on its scientific value. The feedback was overwhelmingly positive, with many complements on the handling of the logistics and the Zoom meeting format. Nevertheless, the feedback did reveal some areas for improvement. 

Active participation:  I just really struggled to participate in the way I would have done had it been in person.

Breakout discussions were an integral part of the original program. The modified format could not include breakout sessions and the became more of a series of preenttions from the projec  rather than a workshop.  missed the whole interaction/face-to-face aspect that this meeting was supposed to be all about... This makes the necessity of follow-up unconference style sessions even more important. 


\section{Conclusions}


\section{Acknowledgements}
I would like to thank, Rubin Observatory communications team for their help with the website and support in converting this workshop to an all-virtual format at short notice,  NOIRLab IT for technical support and for being on call during the workshop in the event of any network problems, SLAC for the use of their Zoom system, and Arjun Dey, NOIRLab and Richard Dubois, SLAC for their advice on running and all-virtual meeting.

\appendix
% Include all the relevant bib files.
% https://lsst-texmf.lsst.io/lsstdoc.html#bibliographies
\section{References} 
\label{sec:bib}
\bibliography{local,lsst,lsst-dm,refs_ads,refs,books}

% Make sure lsst-texmf/bin/generateAcronyms.py is in your path
\section{Acronyms} \label{sec:acronyms}
\input{acronyms.tex}
% If you want glossary uncomment below -- comment out the two lines above
\printglossaries


\end{document}
